% Options for packages loaded elsewhere
\PassOptionsToPackage{unicode}{hyperref}
\PassOptionsToPackage{hyphens}{url}
%
\documentclass[
]{article}
\usepackage{amsmath,amssymb}
\usepackage{lmodern}
\usepackage{ifxetex,ifluatex}
\ifnum 0\ifxetex 1\fi\ifluatex 1\fi=0 % if pdftex
  \usepackage[T1]{fontenc}
  \usepackage[utf8]{inputenc}
  \usepackage{textcomp} % provide euro and other symbols
\else % if luatex or xetex
  \usepackage{unicode-math}
  \defaultfontfeatures{Scale=MatchLowercase}
  \defaultfontfeatures[\rmfamily]{Ligatures=TeX,Scale=1}
\fi
% Use upquote if available, for straight quotes in verbatim environments
\IfFileExists{upquote.sty}{\usepackage{upquote}}{}
\IfFileExists{microtype.sty}{% use microtype if available
  \usepackage[]{microtype}
  \UseMicrotypeSet[protrusion]{basicmath} % disable protrusion for tt fonts
}{}
\makeatletter
\@ifundefined{KOMAClassName}{% if non-KOMA class
  \IfFileExists{parskip.sty}{%
    \usepackage{parskip}
  }{% else
    \setlength{\parindent}{0pt}
    \setlength{\parskip}{6pt plus 2pt minus 1pt}}
}{% if KOMA class
  \KOMAoptions{parskip=half}}
\makeatother
\usepackage{xcolor}
\IfFileExists{xurl.sty}{\usepackage{xurl}}{} % add URL line breaks if available
\IfFileExists{bookmark.sty}{\usepackage{bookmark}}{\usepackage{hyperref}}
\hypersetup{
  pdftitle={Learning R for Wildlife},
  pdfauthor={Jon Weissman jweissma@gmail.com},
  hidelinks,
  pdfcreator={LaTeX via pandoc}}
\urlstyle{same} % disable monospaced font for URLs
\usepackage[margin=1in]{geometry}
\usepackage{graphicx}
\makeatletter
\def\maxwidth{\ifdim\Gin@nat@width>\linewidth\linewidth\else\Gin@nat@width\fi}
\def\maxheight{\ifdim\Gin@nat@height>\textheight\textheight\else\Gin@nat@height\fi}
\makeatother
% Scale images if necessary, so that they will not overflow the page
% margins by default, and it is still possible to overwrite the defaults
% using explicit options in \includegraphics[width, height, ...]{}
\setkeys{Gin}{width=\maxwidth,height=\maxheight,keepaspectratio}
% Set default figure placement to htbp
\makeatletter
\def\fps@figure{htbp}
\makeatother
\setlength{\emergencystretch}{3em} % prevent overfull lines
\providecommand{\tightlist}{%
  \setlength{\itemsep}{0pt}\setlength{\parskip}{0pt}}
\setcounter{secnumdepth}{-\maxdimen} % remove section numbering
\ifluatex
  \usepackage{selnolig}  % disable illegal ligatures
\fi

\title{Learning R for Wildlife}
\author{Jon Weissman
\href{mailto:jweissma@gmail.com}{\nolinkurl{jweissma@gmail.com}}}
\date{4/28/2022}

\begin{document}
\maketitle

{
\setcounter{tocdepth}{2}
\tableofcontents
}
This document compiles R resources that I recommend. R is not the most
straightforward programming language, and different styles of R code can
appear drastically different. Almost every style will look quite
different from more conventional programming languages like C or Python.

Given that, it is easy to head down the wrong path in R. The resources I
list here are largely chosen to avoid those pitfalls.

\begin{center}\rule{0.5\linewidth}{0.5pt}\end{center}

\hypertarget{basics}{%
\subsection{Basics}\label{basics}}

Do you want to learn R?

If you have little to no coding experience and want to get started, or
are new to R and data processing, this online book is great:

\textbf{Statistical Inference via Data Science}
\url{https://moderndive.com/1-getting-started.html}

If you have are kinda familiar with code, or have even coded in old
school base R extensively, but are new to tidy R or data science with R,
\href{https://github.com/hadley}{Hadley's} book is the definitive guide
to analyzing data with R:

\textbf{R for Data Science} \url{https://r4ds.had.co.nz/index.html}

Both of these resources are listed on the
\href{https://www.tidyverse.org/learn/}{tidyverse} website.

Once you've started using these, you'll quickly need more reference info
on tidyverse tools. Thankfully, these are extremely well documented with
vignettes and examples. Many of the most commonly used data manipulation
functions come from the package \texttt{dplyr}:

\textbf{Introduction to dplyr}
\url{https://dplyr.tidyverse.org/index.html}

You'll also want to plot data, and hence learn \texttt{ggplot2}:

\textbf{Learning ggplot2}
\url{https://ggplot2.tidyverse.org/\#learning-ggplot2}

If you want to get started plotting quickly, you can find what you need
here:

\textbf{R Graphics Cookbook} \url{https://r-graphics.org/}

\begin{center}\rule{0.5\linewidth}{0.5pt}\end{center}

\hypertarget{statistics}{%
\subsection{Statistics}\label{statistics}}

Do you want to get some basic descriptive statistics (mean, variance,
etc)? Do you want to make some plots or fit a basic model?

\textbf{R for Data Science}, again, has you covered.

\begin{itemize}
\tightlist
\item
  \href{https://r4ds.had.co.nz/exploratory-data-analysis.html}{Ch 7
  Exploratory Data Analysis}
\item
  \href{https://r4ds.had.co.nz/model-intro.html}{Ch 22 Modeling}
\end{itemize}

Do you want to model your data for legit scientific use? Understand
linear or logistic regression? Use (and actually understand) generalized
additive models?

By far the best resource for statistical data analysis I've ever seen is
\href{http://bactra.org/}{Cosma Shalizi's}:

\textbf{Advanced Data Analysis from an Elementary Point of View}
\url{https://www.stat.cmu.edu/~cshalizi/ADAfaEPoV/ADAfaEPoV.pdf}

Most texts either focus on the principles and theory of statistics, or
the nitty gritty of data analysis code. This book stays focussed on the
basics while also demonstrating everything with actual R code and plots.
The one downside is that it uses base R instead of tidyverse stuff.
However, once you are familiar with tidyverse stuff it is not hard to go
between the two. The largest difference is in the base \texttt{plot}
versus the tidy \texttt{ggplot}.

Some of the content in this book is probably not of interest. I
certainly haven't made it through the whole thing. So I'll highlight
some sections I've found particularly useful here:

\begin{itemize}
\tightlist
\item
  1.1 \emph{Regression: Predicting and Relating Quantitative Features}
\item
  2.4 \emph{Linear Regression Is Not the Philosopher's Stone}
\item
  3 \emph{Model Evaluation}
\item
  8 \emph{Additive Models}
\item
  12 \emph{Generalized Linear Models and Generalized Additive Models}
\end{itemize}

I think if you were to work through that whole book you would have a
better statistics education than the majority of professional data
scientists.

\begin{center}\rule{0.5\linewidth}{0.5pt}\end{center}

\hypertarget{spatial-data}{%
\subsection{Spatial Data}\label{spatial-data}}

Do you want to make maps? Do spatial statistics? Explore movements?

Spatial data comes in two forms: Vector and Raster.

\begin{itemize}
\tightlist
\item
  Vector: This is data associated with precise coordinates. Think
  points, lines, gps fixes, roads.
\item
  Raster: This is gridded data that is mapped to a spatial surface. The
  data is in images, and each pixel of that image covers a general area
  rather than precise coordinates. Think satellite imagery or elevation
  models.
\end{itemize}

For vector data, you will want to use \texttt{sf}:

\textbf{Simple Features for R} \url{https://r-spatial.github.io/sf/}

This is an R package, not a book. However, like many good R packages, it
has excellent documentation in the \emph{Articles} or \emph{vignettes}.

\texttt{sf} is at the heart of all the code I've written to process
telemetry data. I can't recommend it enough, and
\href{https://github.com/edzer}{Edzer} is the most responsive developer
I've encountered.

For more info on how to use \texttt{sf}:

\textbf{Spatial Analysis with R}
\url{https://chiajung-yeh.github.io/Spatial-Analysis/}

This text by \href{https://github.com/chiajung-yeh}{Yeh Chia Jung} looks
nice. It's basically and in-depth how to on using the \texttt{sf}
package. It illustrates how to use \texttt{ggplot} in combination with
spatial data, and also explains what goes into making data spatial.

For raster data, the corresponding package is \texttt{stars}:

\textbf{Spatiotemporal Arrays: Raster and Vector Datacubes}
(\url{https://r-spatial.github.io/stars/})

However, this is a younger package and has a strong competitor in
\href{https://rspatial.org/terra/pkg/index.html}{\texttt{terra}}, which
I haven't used but I know is popular.

More info:

The underlying libraries driving all of these spatial data tools come
from \href{https://www.osgeo.org/partners/ogc/}{Open Geospatial
Consortium}. These tools, like \href{https://gdal.org/}{GDAL} and
\href{https://proj.org/}{PROJ} also power software like
\href{https://www.qgis.org/en/site/}{QGIS} and
\href{https://postgis.net/}{PostGIS}.

The good thing about that is that it's easy to exchange data between any
of these tools, and often an analysis done in one can be easily
replicated in another. For example, I've often done something using the
QGIS graphical interface, and then once I've figured out what I wanted,
copied the steps over into R or Python using the same underlying GDAL
library.

Unfortunately, ESRI does it's own thing and often doesn't play well with
OGC tools.

\begin{center}\rule{0.5\linewidth}{0.5pt}\end{center}

\hypertarget{advanced-r-topics}{%
\subsection{Advanced R topics}\label{advanced-r-topics}}

At some point writing tidy R code you are going to get frustrated and
feel limited with what you can do. Most likely you will then start
trying to write Python using R, with for loops and such. This can work
but will also get really messy, and if you choose this path, you will
regret ever using R because things would have been easier and faster had
you started with Python.

Fortunately there is another path, and once again Hadley is there to
guide us:

\textbf{Advanced R} \url{https://adv-r.hadley.nz/}

\begin{center}\rule{0.5\linewidth}{0.5pt}\end{center}

\hypertarget{rmarkdown}{%
\subsection{RMarkdown}\label{rmarkdown}}

This is an R Markdown document. Markdown is a simple formatting syntax
for authoring HTML, PDF, and MS Word documents. For more details on
using R Markdown see \url{http://rmarkdown.rstudio.com}.

For more info on how to write R Markdown and do really fancy stuff with
it:

\textbf{R Markdown: The Definitive Guide}
\url{https://bookdown.org/yihui/rmarkdown/}

\begin{center}\rule{0.5\linewidth}{0.5pt}\end{center}

\end{document}
